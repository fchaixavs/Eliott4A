\documentclass[
	a4paper,
	twocolumn=false,
	12pts,
	DIV=calc]%
	{scrartcl}
\usepackage{config}

\title{Chapitre 7 : Calcul littéral}
\begin{document}
\maketitle
\section*{Activité du dallage}
\begin{itemize}
	\item Dans cette activité, la proportionnalité ne marche pas.
	\item Cette activité montre l’utilité du calcul avec des lettres (littéral), pour remplir \emph{rapidement} le bon de commande.
	\item Plusieurs formules ont été trouvées et on peut les vérifier avec la largeur 4 (Nombre de dalles : 26)
\end{itemize}

\section{Utiliser des lettres}

\paragraph{Définition} Une expression mathématique dans laquelle figure une ou plusieurs lettres et
où chaque lettre représente un nombre variable est appelée expression littérale.

Ces expressions littérals permettent :

\begin{itemize}
	\item d’établir des formules (voir les formules d’aires)
	\item d’exprimer \og en fonction de\fg
	[figure rectangle : Caption = "Le périmètre du rectangle ABCD en fonction de x est : $pABCD = 7 + 4x + 7 + 4x = 14 + 8x$"]
	\item de résoudre des problèmes
	\begin{itemize}
		\item dallage de la piscine
		\item chapitre 9
	\end{itemize}
\end{itemize}

\section{Transformation d'écritures littérales}
\subsection{Convention}
On peut supprimer le signe de la \emph{multiplication} entre :
\begin{itemize}
	\item 2 lettres : $a \times b = ab$ ou $x \times x = x^2$ ou $x \times x \times x = x^3$
	\item 1 nombre et 1 lettre :\\
		$5 \times y=5 y =y \times 5$\\
		ou $x \times 1=1 \times x=1 x=x$\\
		ou $x \times 0=0 \times x=0 x=0$\\
		ou $x \times −1=−1 \times x=−1 x=−x$\\
	\item 1 nombre et 1 parenthèse : $3 \times (2+x)=3(2+x)$
	\item 2 parenthèses : $(a+b) \times (c +d)=(a+b)(c+ d)$
\end{itemize}

\begin{attention}
Pas de suppression entre 2 nombres !! $7×3 \neq 73$
\end{attention}

\subsection{Réduire une expression littérale}

\paragraph{Définition} Réduire une expression littérale signifie regrouper les termes semblables en suivant les règles de calcul.

\paragraph{Exemple}
\begin{itemize}%[.]%[label=\textbullet]
	\item $2 \times +3 x=5 x$ mais $2 x +3 y$ ne peut pas se réduire.
	\item $4 x−3+7 x^2 −7+9 x^2 −10 x=−6 x−10+16 x^2$
	\item $2 x\times 4 x \times 5=2 \times x \times 4\times x\times 5=2\times 4\times 5\times x\times x=40\times x =40 x^2$
\end{itemize}

\end{document}
