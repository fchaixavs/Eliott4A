\documentclass[
	a4paper,
	twocolumn=false,
	12pts,
	DIV=calc]%
	{article}
\usepackage{config}

\title{Chapitre 7 : Calcul littéral}
\date{}
\begin{document}
\maketitle
\section*{Activité du dallage}
\begin{itemize}
	\item Dans cette activité, la proportionnalité ne marche pas.
	\item Cette activité montre l’utilité du calcul avec des lettres (littéral), pour remplir \emph{rapidement} le bon de commande.
	\item Plusieurs formules ont été trouvées et on peut les vérifier avec la largeur 4 (Nombre de dalles : 26)
\end{itemize}

\section{Utiliser des lettres}

\paragraph{Définition} Une expression mathématique dans laquelle figure une ou plusieurs lettres et
où chaque lettre représente un nombre variable est appelée expression littérale.

Ces expressions littérals permettent :

\begin{itemize}
	\item d’établir des formules (voir les formules d’aires)
	\item d’exprimer "en fonction de"
	[figure rectangle : Caption = "Le périmètre du rectangle ABCD en fonction de x est : $P_{ABCD} = 7 + 4x + 7 + 4x = 14 + 8x$"]
	\item de résoudre des problèmes
	\begin{itemize}
		\item dallage de la piscine
		\item chapitre 9
	\end{itemize}
\end{itemize}

\section{Transformation d'écritures littérales}
\subsection{Convention}
On peut supprimer le signe de la \emph{multiplication} entre :
\begin{description}
	\item 2 lettres : $a \times b = ab$ ou $x \times x = x^2$ ou $x \times x \times x = x^3$
	\item 1 nombre et 1 lettre :\\
		$5 \times y=5 y =y \times 5$\\
		ou $x \times 1=1 \times x=1 x=x$\\
		ou $x \times 0=0 \times x=0 x=0$\\
		ou $x \times −1=−1 \times x=−1 x=−x$\\
	\item 1 nombre et 1 parenthèse : $3 \times (2+x)=3(2+x)$
	\item 2 parenthèses : $(a+b) \times (c +d)=(a+b)(c+ d)$
\end{description}

\begin{attention}
Pas de suppression entre 2 nombres !! $7×3 \neq 73$
\end{attention}

\subsection{Réduire une expression littérale}

\paragraph{Définition} Réduire une expression littérale signifie regrouper les termes semblables en suivant les règles de calcul.

\paragraph{Exemple}
\begin{itemize}%[.]%[label=\textbullet]
	\item $2 \times +3 x=5 x$ mais $2 x +3 y$ ne peut pas se réduire.
	\item \begin{math}
	4 x−3+7 x^2 −7+9 x^2 −10 x\\
	=−6 x−10+16 x^2
	\end{math}
	\item 
		\begin{math}
		2 x\times 4 x \times 5 \\
		=2 \times x \times 4\times x\times 5\\
		=2\times 4\times 5\times x\times x\\
		=40\times x \\
		=40 x^2
		\end{math}
	\item
		\begin{math}
		3 x \times 2x^2 \\
		=3\times 2\times x\times x\\
		=6 \times x 3
		\end{math}
\end{itemize}

\begin{attention}
Attention !!! 3 x +2 x 2 ne se réduit pas !
\end{attention}

\subsection{Calcul de la valeur d’une expression littérale}

Calculer la valeur d’une expression littérale, c’est remplacer la lettre par une valeur
numérique. Il est souvent plus judicieux de coisir la forme la plus réduite avant de faire les
calculs..

\paragraph{Exemples}
\begin{itemize}
	\item
	Soit $A=2+5 x +8−12 x +2 x^2$ Calculer A pour $x=-3$\\On déduit :\\
		\begin{math}
		A=2+5 x +8−12 x +2 x^2 \\
		=10−7 x+2 x^2
	\end{math}
	\item Pour $x=-3$\\ 
		\begin{math}
		A=10−7×(−3)+2×(−3)\\
A=10−7×(−3)+2×9\\
A=10−(−21)+18\\
A=10+21+18=49
		\end{math}
\end{itemize}

\section{Développement}

\subsection{Calcul de la valeur d’une expression littérale}

Calculer la valeur d’une expression littérale, c’est remplacer la lettre par une valeur
numérique. Il est souvent plus judicieux de coisir la forme la plus réduite avant de faire les
calculs.

\paragraph{propriété}\emph{addition et parenthèses.}\\
Dans une suite d'\emph{additions} et de \emph{soustractions}, on peut supprimer des parenthèses précédées d'un signe <<$+$>> en conservant les signes des termes \emph{intérieurs} aux parenthèses.

\paragraph{exemple} 
\begin{math}
7x + (6-2xy)+(-4x+3y)\\
= 7x+6-2xy-4x+3y\\
= 3x+6-2xy+3y
\end{math}

\paragraph{Propriété}\emph{soustraction et parenthèses}\\
Dans une suite d'additions et de soustractions, on peut supprimer des parenthèses précédées par un signe <<$-$>> en \emph{changeant} les signes des termes \emph{intérieurs} aux parenthèses.

\paragraph{Exemples:}
\begin{itemize}
	\item
		\begin{math}
		3x-(x+3) \\
		= 3x-x-3 \\
		= 2x-3
		\end{math}
	\item
		\begin{math}
		9x^2+4-(-2x^2+4x-5) \\
		= 9x^2+4+2x^2-4x+5 \\
		= 11x^2+9-4x
		\end{math}
\end{itemize}

\paragraph{Propriété:}\emph{la double distributivité}\\
(Voir 7 c) P.28)\\
Pour tous les nombres a, b, c et d :\\
	\begin{math}
	(a+b) \times (c+d) \\
	= a \times c + a \times d + b\times c + b\times d
	\end{math}
\paragraph{Exemples}
\begin{itemize}
\item
\begin{math}
	(3x + 4)(x+2)\\
	= 3x\times x + 3x\times 2 + 4x + 4\times 2\\
	= 3x^2 + 6x + 4x + 8\\
	= 3x^2 + 10x + 8
\end{math}
\item
\begin{math}
	(2x-4)(-3x+7)\\
	= 2x \times -3x + 2x \times 7 -4 \times (-3x) -4\times 7\\
	= -6x^2 + 14x + 12x -28\\
	=-6x^2 + 26x - 28
\end{math}
\end{itemize}

\section{Résolution de problèmes (avec tableur)}
\subsection{Piscine et dallage}
\subsection{Programme de calcul}

\begin{enumerate}
	\item Choisir un nombre
	\item Lui ajouter 5
	\item Multiplier le résultat par 4
	\item Soustraire 20 au dernier résultat obtenu
\end{enumerate}

\paragraph{1er essai}pour le nombre 13\\
\begin{itemize}
	\item $13$
	\item $13+5=18$
	\item $18\times 4 = 72$
	\item $72-20=52$
\end{itemize}

\paragraph{2ème essai}pour le nombre 2\\
\begin{math}
	[(2+5)\times 4]-20\\
	=7\times 4 -20\\
	=28-20\\
	=8
\end{math}

\paragraph{Conjecture (=hypothèse, pronostic)~:}"On obtient le quadruple du nombre de départ"

\paragraph{Démonstration}
Soit $x$ le nombre de départ.\\
\begin{math}
(x+5)\times 4 - 20\\
= 4\times x + 4\times 5 - 20\\
= 4x + 20 - 20\\
=4x
\end{math}
\\→ La conjecture est démontrée.
\end{document}
